\documentclass[conference]{IEEEtran}
\IEEEoverridecommandlockouts
% The preceding line is only needed to identify funding in the first footnote. If that is unneeded, please comment it out.
\usepackage{cite}
\usepackage{amsmath,amssymb,amsfonts}
\usepackage{algorithmic}
\usepackage{graphicx}
\usepackage{textcomp}
\usepackage{xcolor}
\def\BibTeX{{\rm B\kern-.05em{\sc i\kern-.025em b}\kern-.08em
    T\kern-.1667em\lower.7ex\hbox{E}\kern-.125emX}}
\begin{document}

\title{CYBER RISK AND CYBER INSURANCE\\}

\author{\IEEEauthorblockN{Suraj Eswaran}
\IEEEauthorblockA{\textit{Colorado State University} \\
\textit{Fort Collins,Colorado}
\\suraj22@colostate.edu}}

\maketitle

\begin{abstract}
Cyber risk is a form of risk from the exposure resulting from a cyber-attack or data breach. Organizations tend to become more vulnerable to these kinds of threats due to their high reliability on computers, networks, and information in order to get along with the delivery of the services. In case of failure in these systems, these organization will face a negative impact in the processes , which will create a negative impact on the organization. So, this paper deals with the understanding the various views on cyber risk insurance and its challenges that arises in insurance markets in the recent years.   
\end{abstract}

\begin{IEEEkeywords}
Cyber risk, Cyber risk Insurance, 
\end{IEEEkeywords}

\section{Motivation}
Living in the digital age means cyberterrorists have opportunities to exploit the information of individuals, government institutions, and even larger organization. Due to the effect of covid-19 pandemic, the new opportunities for cybersecurity risk areas are emerging. In order to figure it out, it is necessary to perform an empirical strategy for cyber risk assessment which would be helpful in concluding with perspective of whether cyber risk to be covered by capital management or risk-based approach.  

\section{PROPOSED APPROACH}
The study explores the different views of cyber risk that has emerged in the recent years. First approach deals with the study of Operational Risk which is risk that are caused by the actual losses due to inappropriate procedures, systems or policies followed by the organizations\cite{b1} \cite{b2}. It focuses on how things are accomplished in an organization and not on what is produced from that organization. So, operational risk can be utilized as a technique to get along with this exposure.

Second approach deals with utilization of a mechanism for avoiding cyber risk that seeks to remove the possibility of activities that creates risk\cite{b3}. Cyber risk insurance is a tool that are useful to compensate the loss from the cyber risk incidents\cite{b3}\cite{b4}. For this study, we are preparing the various components for an efficient cyber risk management by analyzing internal risk management with risk awareness factors.  


\section*{References}

Please number citations consecutively within brackets \cite{b1}. The 
sentence punctuation follows the bracket \cite{b2}. Refer simply to the reference 
number, as in \cite{b3}---do not use ``Ref. \cite{b3}'' or ``reference \cite{b3}'' except at 
the beginning of a sentence: ``Reference \cite{b3} was the first $\ldots$''


\begin{thebibliography}{00}
\bibitem{b1} Romanosky, S., Ablon, L., Kuehn, A., and Jones, T. (2019). Content analysis of cyber insurance policies: how do carriers price cyber risk?. Journal of Cybersecurity, 5(1), tyz002.
\bibitem{b2} Falco, G., Eling, M., Jablanski, D., Miller, V., Gordon, L. A., Wang, S. S., ... and Donavan, E. (2019). A research agenda for cyber risk and cyber insurance. In Workshop on the Economics of Information Security (WEIS).
\bibitem{b3} Mukhopadhyay, A., Chatterjee, S., Saha, D., Mahanti, A., and Sadhukhan, S. K. (2013). Cyber-risk decision models: To insure IT or not?. Decision Support Systems, 56, 11-26.
\bibitem{b4} Camillo, M. (2017). Cyber risk and the changing role of insurance. Journal of Cyber Policy, 2(1), 53-63.

\end{thebibliography}

\section{APPLICABLE JOURNALS}
\begin{itemize}
    \item Oxford Journal of Cybersecurity
    \item Journal of Information Assurance and Cybersecurity
    \item The Journal of Cyber Security and Information Systems
    \item Journal of Cyber Security Technology

\end{itemize}

\section{CONFERENCES}
\begin{itemize}
    \item International Conference on Cyber Security and Internet of Things
\item International Conference on Cyber Security and Protection of Digital Services(Cyber Security)
\item 	International Conference on Information Systems Security and Privacy
\item	RSA Conference
\end{itemize}

\section{NAME OF THE ORGANIZATIONS/RESEARCH GROUPS}
\begin{itemize}
    \item Cyber Defense Labs
    \item Center for Internet Security
\item	National Council for ISACs
\item CyberCube
\item	RiskLens
\end{itemize}

\section{INDUSTRIAL PUBLICATIONS}
\begin{itemize}
    \item Krebs on Security
    \item TaoSecurity
    \item Threat Post
\end{itemize}


\section{NEWS SOURCES}
\begin{enumerate}
    \item https://www.propertycasualty360.com/2020/10/02/pandemic-threats-top-cyber-risks-as-the-biggest-threat-for-businesses-in-2020/?slreturn=20200905120520 
    \item https://www.insurancebusinessmag.com/us/news/cyber/cyber-risk-may-eventually-surpass-insurance-industrys-capacity--experts-235096.aspx 
\end{enumerate}
\end{document}
